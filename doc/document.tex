\documentclass{article}
  \parindent = 0mm % Sin sangría
  \usepackage[utf8]{inputenc}
  \usepackage[T1]{fontenc}
  \usepackage[spanish]{babel}
  \usepackage{graphicx}
  \usepackage{amstext}
  \usepackage{amsmath}
  \usepackage{booktabs}
  \usepackage{subfigure}
  \usepackage{footnote}
  \usepackage{hyperref}
  \usepackage{algpseudocode,algorithm,algorithmicx}
  \usepackage[font=small,labelfont=bf]{caption}

\begin{document}
  \begin{center}
    {\sc \large Enfoque de pulsos}
    
    {\sc \large Proyecto 2020}
    \linebreak

    {\rm Joaquín Correa - \today}
  \end{center}
  \section*{Introducción}

  Como proyecto de aplicación del algorítmo del Pulso sugiero su utilización al problema de ruteo y decisión de construcción de ciclovías en una ciudad. Dado una ciudad y un conjunto de puntos orígenes y destinos con una demanda asociada.

  Se utiliza el pulso como método alternativo a la programación matemática, ya que al ser en parte un problema de decisión y por lo tanto un MILP, es sabido que no siempre escala bien con instancias de datos grandes.

  Finalmente se realiza un comparación en medida de lo posible con el MILP y se incluye una sección con características propias y mejoras futuras del trabajo.

  \section*{Problema}
  El problema consiste en decidir dónde construir ciclovías de manera satisfaga las necesidades de los usuarios, representados por la demanda, de la mejor manera. Matemáticamente, el modelo consiste en un grafo dirigido y un conjunto de pares origen-destino donde para cada uno se tiene un valor de demanda (por ejemplo cantidad de personas). Para cada arco del grafo, se tiene un costo de usuario que corresponde al costo percibido por un usuario si lo utiliza para desplazarse y un costo de construcción de diversos tipos de ciclovías. Al construir una ciclovía sobre un arco, el costo de usuario disminuye. Finalmente se tiene un valor de presupuesto que limita la construcción de ciclovías.

  La formulación matemática se muestra:

  \begin{align}
    \text{min} & \sum_{k \in O} \sum_{i \in I} \sum_{e \in E} x_{ek}^i c_e^i & \label{eq:objective} \\
    \text{s.t.} & \sum_{i \in I}\sum_{e \in E_n^+} x_{ak}^i - \sum_{i \in I}\sum_{e \in E_n^-} x_{ak}^i = b_{nk} & \forall k \in O, n \in V \label{eq:flowconservation} \\
          & \sum_{i \in I} y_{e}^i \leq 1 & \forall e \in E \label{eq:onlyoneinfraactive} \\
          & \sum_{i \in I} \sum_{e \in E} m_e^i y_e^i \leq B & \label{eq:respectbudget} \\
          & x_{ek}^i \leq y_e^i  D_k & \forall e \in E, i \in I, k \in O \label{eq:restrictflowbyinfras} \\
          & x_{ek}^i \geq 0, y_e^i \in \{0,1\} &
  \end{align}

  Donde:
  \begin{description}
    \item[$O$]: Conjunto de pares origen-destino.
    \item[$I$]: Conjunto de infraestructuras.
    \item[$G=(V, E)$]: Grafo dirigido, compuesto por el conjunto de vértices $V$ y arcos $E$.
    \item[$c_e^i$]: Parámetro, costo de usuario de atravesar el arco $e$ utilizando la infraestructura $i$.
    \item[$b_{nk}$]: Parámetro, valor de la ecuación de conservación del flujo. Vale $D_k$ si $n$ es el origen de $k$, $-D_k$ si es el destino y cero en otro caso.
    \item[$m_e^i$]: Parámetro, costo de construcción de la infraestructura $i$ sobre el arco $e$.
    \item[$B$]: Parámetro, valor del presupuesto total de construcción.
    \item[$x_{ek}^i$]: Es la variable mayor a cero que modela el flujo del par origen-destino $k$, sobre el arco $e$ utilizando la infraestructura $i$.
    \item[$y_e^i$]: Es la variable binaria que indica si la infraestructura $i$ está activa en el arco $e$.
  \end{description}

  Y las ecuaciones:

  \begin{description}
    \item[(\ref{eq:objective})] Objetivo, se minimiza el costo total percibido por el flujo.
    \item[(\ref{eq:flowconservation})] Restricción de conservación del flujo.
    \item[(\ref{eq:onlyoneinfraactive})] Restricción que permite solo una infraestructura activa por arco.
    \item[(\ref{eq:respectbudget})] Restricción de presupuesto.
    \item[(\ref{eq:restrictflowbyinfras})] Restricción que permite al flujo utilizar la infraestructura activa únicamente.
  \end{description}

  \subsection*{Modelado}

  Se puede notar de la formulación, que si bien el grafo base es un grafo simple, se está trabajando sobre un multigrafo. Cada arco del grafo base es replicado tantas veces como cantidad de infraestructuras hay, y cada replica tiene un costo de usuario y costo de construcción asociado.

  Lo que se busca entonces, son caminos óptimos por este multigrafo, con restricción de presupuesto, que determina para cada par de nodos, que arco se elije, si se elije alguno.

  \subsection*{Dificultades}

  Este problema, asi como fué formulado puede resolverse en un MILP solver para grafos no muy grandes de manera conveniente. Pero, al ser en parte un problema de programación entera sabemos es un problema contenido en $NP$ y por lo tanto, todavía, difícil de escalar.

  Si particionamos, para el análisis, el problema en dos, por un lado esta la decisión de cuál y dónde construir infraestructura y por otro la de encontrar el camino más corto para cada par origen-destino. Obviamente estos subproblemas no pueden ser tratados por separado en este caso. La primera partición del problema es la difícil, la segunda, vista de manera aislada es un problema \it{facil}. El flujo entre dos pares origen destino no esta relacionado con otros flujos, ni los arcos tienen capacidad. Por lo tanto, la segunda partición es simplemente el problema del camino mas corto entre dos puntos, repetido $card(O)$ veces. Lo que se busca en este trabajo, es utilizar el pulso para resolver ambos problemas al mismo tiempo.

  \section*{Aplicación del Pulso}



  TODO: parrafo en que se explica la complejidad de que hayan varios od.
  Supongamos, para simplificar, que tenemos un único par origen-destino.
  

  Descripción del algorítmo del pulso.

  \subsection*{Caso de estudio}

  Presentacion del grafo, origenes y destinos.

  \section*{Resultados}

  Comparación de resultados.
  Comparar ejecución con glpk y cbc.

  \section*{Implementacion}

  \subsubsection*{Consideraciones}

  Optimizaciones, desiciones de diseño, configuración.

\end{document}
